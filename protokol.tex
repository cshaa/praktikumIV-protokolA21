% !TEX program = xelatex

% Nejlepší zážitek zaručí:
%
% TeX distribuce: texlive-full
%
% Editor:
%   VS Code s doplňky
%       * LaTeX Workshop
%       * LaTeX Utilities
%       * Gnuplot
%
% Další závislosti:
%   latexmk
%   bibtex
%   gnuplot


% Jak používat:
% Zkompilovat: make
% Gnuplot: make gnuplot
% Vyčistit: make clean


% Základní balíčky
\documentclass[10pt,a4paper]{article}
\usepackage[utf8]{inputenc}
\usepackage[czech]{babel}
\usepackage{graphicx}
\usepackage{lmodern}
\usepackage{amsmath}
\usepackage{hyperref}
\usepackage{gensymb}
\usepackage[top = 1.5cm, bottom = 1.5cm, left = 1.5cm, right = 1.5cm]{geometry}

% Bibtex
\usepackage{etoolbox}
\patchcmd{\thebibliography}{\section*{\refname}}{}{}{}

% Pro titulní stránku
\usepackage{titlesec}
\usepackage{setspace}
\usepackage{framed}
\usepackage{array}

% Vlastní balíčky 
\usepackage{gnuplottex}
\usepackage{epstopdf}
\usepackage{csvsimple}


\renewcommand{\U}[1]{\ensuremath{\,\mathrm{#1}}}
\newcommand{\°}{\degree}

\newcommand{\titjmeno}{Michal Grňo}
\newcommand{\titobor}{FOF}


\newcommand{\titcislo}{A20}
\newcommand{\titnazev}{Fourierovská infračervená spektroskopie}
\newcommand{\titmereni}{4. 11. 2019}
\newcommand{\titodevzdani}{11. 11. 2019}


\begin{document}


\thispagestyle{empty}
\newgeometry{top = 2.5cm, bottom = 0cm, left = 2.5cm, right = 3cm}

{%T tomto je uzavřena celá titulka
%Tloušťka rámečku
\setlength{\fboxrule}{1.5pt}

\noindent
\framebox{
\begin{minipage}{\textwidth}
\setlength{\parindent}{17.62482 pt}
\phantom{d}

\begin{minipage}{0.6\textwidth}
{
\Large Kabinet výuky obecné fyziky, UK MFF\\
}
\vspace*{0.2cm}

{
\bfseries
\huge Fyzikální praktikum %ČÍSLO
}
\end{minipage}
\begin{minipage}{0.4\textwidth}
\begin{center}
\includegraphics[width=4.5cm]{ZFP.jpg}
\end{center}
\end{minipage}\\\\

%\vspace*{0.5cm}

{
\setstretch{1.5}
\Large
\noindent
Úloha č. \titcislo

\noindent
Název úlohy: \titnazev

\noindent
Jméno: \titjmeno
\hspace*{\fill}
Obor: \titobor

\noindent
Datum měření: \titmereni
\hspace*{\fill}
Datum odevzdání: \titodevzdani

\phantom{d}
}
\end{minipage}
}
%Konec horního rámečku

{
\phantom{d}

\Large
Připomínky opravujícího:\\
\vspace*{6.75cm}
}

\newcommand{\linka}{\noalign{\hrule height 1pt}}
\newcommand{\linkadva}{\noalign{\hrule height 1.5pt}}
\setlength\extrarowheight{9.5pt}
\Large
\noindent
\begin{tabular}{!{\vrule width 1.5pt} l !{\vrule width 1pt} c !{\vrule width 1pt} c !{\vrule width 1.5pt}}
\linkadva
   & Možný počet bodů & Udělený počet bodů \\\linkadva
  Práce při měření & 0-3 &  \\\linka
  Teoretická část & 0-2 &  \\\linka
  Výsledky a zpracování měření & 0-9 &  \\\linka
  Diskuse výsledků & 0-4 &  \\\linka
  Závěr & 0-1 &  \\\linka
  Použitá literatura & 0-1 &  \\\linkadva
  \hspace*{\fill} \textbf{Celkem} \hspace*{\fill}& max. 20 &  \\
\linkadva
\end{tabular}
\phantom{d}

Posuzoval: \hspace*{\fill}dne:~~~~~~~~~~~~~~~~~

}%Konec uzavření titulky
\newpage
\newgeometry{top = 2cm, bottom = 2cm, left = 2cm, right = 2cm}
\setcounter{page}{1}

\section{Pracovní úkoly}
\begin{enumerate}
    \item Proměřte rotačně-vibrační absorpční spektrum oxidu uhelnatého ve spektrální oblasti 2000 – 2500 $\U{cm^{-1}}$. Polohy absorpčních pásů zpracujte graficky a lineární regresí určete parametry vystupující v modelu pružného rotátoru pro základní vibrační stav molekuly a první excitovaný vibrační stav. Z těchto parametrů určete vzdálenosti jader uhlíku a kyslíku v základním a prvním excitovaném vibračním stavu.
    \item Spočtěte teplotní a tlakové rozšíření absorpčních pásů, určete rozdíl vibrační frekvence pro isotopomery $^{12}$C$^{16}$O a $^{13}$C$^{16}$O. Porovnejte tyto hodnoty s rozlišením použitého spektrálního přístroje.
    \item Změřte spektrum bez vzorku, určete oblasti absorpce oxidu uhličitého a vodních par v optické dráze spektrometru. Interpretujte nejvýraznější pásy absorpce CO$_2$.
    \item Proměřte spektra propustnosti polyetylénové a polypropylénové folie a interpretujte nejvýraznější pásy.
    \item Proměřte spektra propustnosti a odrazivosti skleněné a safírové destičky. Diskutujte rozdíl mezi oběma vzorky.
\end{enumerate}


\section{Teoretická část}
Infračervená spektroskopie je nástroj používaný k identifikaci molekul přítomných v látce podle módů jejich mechanických kmitů. Konkrétně Fourierovská infračervená spektroskopie detekuje změny ve spektru průchozího elektromagnetického záření způsobené interakcí s kmitající molekulou.

U jednoduchých molekul lze v jejich spektru pozorovat tři charakteristické oblasti –
nízkofrekvenční\footnote{Nízkofrekvenční pás P bude v nižších hodnotách $f \propto \tilde{\nu}$, v grafu to ovšem znamená, že bude vpravo\dots} pás P odpovídající oblasti, kde se rotační energie a energie fotonu skládají, střední pás Q odpovídající vibračním přechodům beze změny rotačního stavu (a kvantového čísla $J$), vysokofrekvenční\footnote{\dots pás R bude naopak mít vysoké hodnoty $\tilde\nu$ a v grafu bude vlevo.} oblast R, kde energie fotonů přispívá jak vibračním, tak rotačním stavům. \cite{studijni-text}

Z vlnočtu pásů, který naměříme pomocí spektroskopie, lze snadno vypočítat jejich energii:
\begin{equation}
    E = \mathrm{h} \mathrm{c} \tilde\nu,
    \label{energie}
\end{equation}
kde $\mathrm{h}$ je Planckova konstanta, $\mathrm{c}$ rychlost světla a $\tilde\nu = \lambda^{-1}$ vlnové číslo. U dvouatomové molekuly pro energii odpovídající pásům $P$ a $R$ při stavu $J$ v základním stavu platí vztah:
\begin{equation}
    \frac{ R_J  - P_J }{ 2J + 1 } \; \frac{1}{\mathrm{h}}
    =  \left( 2B_1 - 3D_1 \right)  - D_1  \left( 2J+1 \right)^2,
    \label{regrese-zakladni}
\end{equation}
a pro molekuly v excitovaném stavu:
\begin{equation}
    \frac{ R_{J-1}  - P_{J+1} }{ 2J + 1 } \; \frac{1}{\mathrm{h}}
    =  \left( 2B_0 - 3D_0 \right)  - D_0  \left( 2J+1 \right)^2,
    \label{regrese-excitovany}
\end{equation}
kde $B_{0,1}$ a $D_{0,1}$ jsou konstanty charakteristické pro danou molekulu \cite{studijni-text}. Pokud známe závislosti $R_J(J)$ a $P_J(J)$, tyto konstanty lze určit lineární regresí z rovnic \eqref{regrese-zakladni} a \eqref{regrese-excitovany}.

\section{Výsledky měření}

\subsection{CO}


\begin{figure}[p]
    \centering
    \begin{gnuplot}[terminal=epslatex,terminaloptions={color size 18cm, 8cm}]
        data = 'data/CO_GRNO.PRN'
        set xrange [2500:2000] reverse
        set xlabel "$\\tilde\\nu \\; [\\U{cm^{-1}}]$"
        set ylabel "$T$"
        plot \
            data using ($1 >= 2385 ? $1 : 1/0):2 w l lc black notitle, \
            data using (2385 >= $1 && $1 >= 2349 ? $1 : 1/0):2 w l t "$CO_2, P$", \
            data using (2349 >= $1 && $1 >= 2290 ? $1 : 1/0):2 w l t "$CO_2, R$", \
            data using (2290 >= $1 && $1 >= 2238 ? $1 : 1/0):2 w l lc black notitle, \
            data using (2238 >= $1 && $1 >= 2143 ? $1 : 1/0):2 w l t "$CO, P$", \
            data using (2143 >= $1 && $1 >= 2033 ? $1 : 1/0):2 w l t "$CO, R$", \
            data using (2033 >= $1 ? $1 : 1/0):2 w l lc black notitle
    \end{gnuplot}
    \caption{Měření transmitance CO.}
    \label{co-trans}
\end{figure}

\begin{figure}[p]
    \centering
    \begin{gnuplot}[terminal=epslatex,terminaloptions={color size 15cm,10cm}]

        load 'matlib.cfg'
        h = 6.62607015e-34
        c = 299792458

        LF_File = 'data/co_tab_grno.txt'
        LF_Columns = 6
        load 'loadfile.cfg'
        N = LF_Rows - 1

        stred_i = -1

        array Energie[N]
        do for [i = 1:N] {
            Energie[i] = real(LF_Col1[i+1]) * 100 * h * c
            if (stred_i == -1 && Energie[i] >= 214500*h*c) {
                stred_i = i
            }
        }

        array R[stred_i - 1]
        array P[N - stred_i + 1]
        do for [i = 1:N] {
            if (i < stred_i) { R[stred_i - i] = Energie[i] }
            else { P[i - stred_i + 1] = Energie[i] }
        }

        set multiplot layout 2,1

        set yrange [4:4.5]
        set ytics 0.1
        set xlabel "$J$"
        set ylabel "$P_J, R_J \\; [10^{-20} \\U{J}]$"
        set key outside
        set key right center

        plot sample \
            [i=1:|P|:1] '+' using (i):(P[i]*1e20) pt 5 t "$P_J$", \
            [i=1:|R|:1] '+' using (i-1):(R[i]*1e20) pt 4 t "$R_J$"


        YSCALE = 1 / 1.e11;

        N = min(|R|+1, |P|)

        array X1[N]
        array Y1[N]
        array Y1_scaled[N]

        set print 'fit1.dat.tmp'
        f1(x) = a1 * x + b1

        do for [i=2:N] {
            X1[i] = (2*i-1)**2
            Y1[i] = (R[i-1] - P[i]) / ( (2*i-1) * h)
            Y1_scaled[i] = Y1[i] * YSCALE

            print sprintf("%f %f", X1[i], Y1[i])
        }

        fit f1(x) 'fit1.dat.tmp' via a1, b1
        f1_scaled(x) = f1(x) * YSCALE


        M = min(|R|+2, |P|-1)

        array X2[M]
        array Y2[M]
        array Y2_scaled[M]

        set print "fit2.dat.tmp"
        f2(x) = a2 * x + b2

        do for [i=3:M] {
            X2[i] = (2*i-1)**2
            Y2[i] = (R[i-2] - P[i+1]) / ( (2*i-1) * h)
            Y2_scaled[i] = Y2[i] * YSCALE

            print sprintf("%f %f", X2[i], Y2[i])
        }

        fit f2(x) 'fit2.dat.tmp' via a2, b2
        f2_scaled(x) = f2(x) * YSCALE


        set xtics rotate
        set yrange [-1.15:-1.12]
        set ytics   0.01
        set xlabel "$\\left(2J + 1\\right)^2$"
        set ylabel "$\\left(P_a - R_b\\right)/\\mathrm{h}\\left(2J + 1\\right)$\n$[10^{11}\\U{s^{-1}}]$" font ",40"
        plot sample \
            [i=2:N:1] '+' using (X1[i]):(Y1_scaled[i]) pt 5 lc 1 notitle, \
            [i=3:M:1] '+' using (X2[i]):(Y2_scaled[i]) pt 4 lc 2 notitle, \
            f1_scaled(x) lc 1 notitle, f2_scaled(x) lc 2 notitle, \
            '+' using (1/0) w lp pt 5 lc 1 t "$R_J - P_J$", \
            '+' using (1/0) w lp pt 4 lc 2 t "$R_{J-1} - P_{J+1}$"



        b1_exp = exponent(b1)
        b2_exp = exponent(b1)
        a1_exp = exponent(a1)
        a2_exp = exponent(a2)

        set print "fitCO.csv.tmp"
        print sprintf("coleqn,colvalue,colstdev")
        print sprintf("$2B_1 - 3D_1 \\; [10^{%d} \\U{s^{-1}}]$,%.4f,%.4f", b1_exp, manexp(b1,-b1_exp), manexp(b1_err,-b1_exp))
        print sprintf("$2B_1 - 3D_1 \\; [10^{%d} \\U{s^{-1}}]$,%.4f,%.4f", b2_exp, manexp(b2,-b2_exp), manexp(b2_err,-b2_exp))
        print sprintf("$D_1 \\;\\; [10^{%d} \\U{s^{-1}}]$,%.2f,%.2f", a1_exp, manexp(-a1, -a1_exp), manexp(a1_err, -a1_exp))
        print sprintf("$D_0 \\;\\; [10^{%d} \\U{s^{-1}}]$,%.2f,%.2f", a2_exp, manexp(-a2, -a2_exp), manexp(a2_err, -a2_exp))

        B1 = .5*(b1 - 3*a1)
        B1_err = .5*sqrt( b1_err**2 - (3*a1_err)**2 )

        B0 = .5*(b2 - 3*a2)
        B0_err = .5*sqrt( b2_err**2 - (3*a1_err)**2 )
        
        print sprintf("$B_1$,%.0f,%.0f", B1, B1_err)
        print sprintf("$B_0$,%.0f,%.0f", B0, B0_err)

    \end{gnuplot}
    \caption{Energie pásů P a R v CO.}
    \label{co-fit}
\end{figure} 

Jako první vzorek jsme měřili transmitanci kyvety se stěnami z CaF$_2$ obsahující CO pod tlakem $7 \U{mbar}$. Měření bylo předem kalibrováno na transmitanci vzduchu a probíhalo na rozsahu $2000$ až $2500 \U{cm^{-1}}$ při rozlišení $0,35 \U{cm^{-1}}$. Protože se mezi kalibrací a měřením změnil podíl CO$_2$ v měřeném prostoru, bylo kromě spektra CO viditelné i spektrum CO$_2$. V obou případech byla jasně rozlišitelná pásma P a R, pásmo Q nebylo pozorováno. Zatímco v případě CO jsou jasně rozlišitelná absorpční maxima odpovídající pásům $P_J$ a $R_J$ pro celočíselné hodnoty $J$, spektrum CO$_2$ vykazuje určitou míru překryvu a jeho maxima nejsou jasně oddělená. Naměřené spektrum bylo zaneseno do grafu na obrázku č. \ref{co-trans}.

Naměřená absorpční maxima jsme potom odstředu očíslovali a jejich vlnočet jsme převedli na energii podle vzorce \eqref{energie}. Pás P začíná od jedničky $P_1 = 4.288 \cdot 10^{-20} \U{J}$ a pokračuje k vyšším hodnotám (v grafu doleva), pás R začíná od nuly $R_0 = 4.280 \cdot 10^{-20} \U{J}$ a pokračuje k nižším hodnotám (v grafu doprava). Takto vypočtené a očíslované energie $P_J$ a $R_J$ jsme v závislosti na jejich kvantovém čísle $J$ vynesli do grafu na obrázku č. \ref{co-fit} nahoře.

Následně jsme podle vzorců \eqref{regrese-zakladni} a \eqref{regrese-excitovany} vynesli závislost $\left(P_a - R_b\right)/\mathrm{h}\left(2J + 1\right)$ na $\left(2J + 1\right)^2$ do grafu na obrázku č. \ref{co-fit} dole. Pomocí lineární regrese metodou nejmenších čtverců jsme nalezli konstanty $B_0, B_1, D_0, D_1$.

\begin{table}[h!]
    \centering
    \begin{tabular}{r|r|r}
        \bfseries A &
        \bfseries hodnota &
        \bfseries směr. odch.
        \csvreader[ head to column names ]{fitCO.csv.tmp}{}
        {
            \csviffirstrow{\\\hline}{\\}
            \coleqn & \colvalue & \colstdev
        }
    \end{tabular}
    \caption{Konstanty charakteristické pro molekulu CO získané regresí}
    \label{tab-co}
\end{table}


\begin{figure}[p]
    \centering
    \begin{gnuplot}[terminal=epslatex,terminaloptions={color size 18cm, 8cm}]
        data = 'data/VZDUCH_GRNO.PRN'
        set key left top
        set xrange [4500:500] reverse
        set xlabel "$\\tilde\\nu \\; [\\U{cm^{-1}}]$"
        set ylabel "$T$"
        plot \
            data using ($1 >= 4000 ? $1 : 1/0):2 w l lc black notitle, \
            data using (4000 >= $1 && $1 >= 3500 ? $1 : 1/0):2 w l t "$H_2O$, 1. h." lc 'green', \
            data using (3500 >= $1 && $1 >= 2400 ? $1 : 1/0):2 w l lc black notitle, \
            data using (2400 >= $1 && $1 >= 2240 ? $1 : 1/0):2 w l t "$CO_2$, 1. h.", \
            data using (2240 >= $1 && $1 >= 2050 ? $1 : 1/0):2 w l lc black notitle, \
            data using (2050 >= $1 && $1 >= 1330 ? $1 : 1/0):2 w l t "$H_2O$, 2. h.", \
            data using (1330 >= $1 && $1 >= 705  ? $1 : 1/0):2 w l lc black notitle, \
            data using (705 >= $1 && $1 >= 670.5 ? $1 : 1/0):2 w l t "$CO_2, P$, 2. h." lc 'red', \
            data using (670.5 >= $1 && $1 >= 666 ? $1 : 1/0):2 w l t "$CO_2, Q$, 2. h.", \
            data using (666 >= $1 && $1 >= 620 ? $1 : 1/0):2 w l t "$CO_2, R$, 2. h.", \
            data using (620 >= $1 ? $1 : 1/0):2 w l lc black notitle
    \end{gnuplot}
    \caption{Měření transmitance vzduchu}
    \label{vzduch-trans}
\end{figure}

\begin{figure}[p]
    \centering
    \begin{gnuplot}[terminal=epslatex,terminaloptions={color size 18cm, 8cm}]

        data1 = 'data/POL1_GRNO.PRN'
        data2 = 'data/POL2_GRNO.PRN'
        data3 = 'data/POL3_GRNO.PRN'
        dataV = 'data/VZDUCH_GRNO.PRN'

        set key left bottom
        set xrange [4000:500] reverse
        set yrange [0:1.1]
        set xlabel "$\\tilde\\nu \\; [\\U{cm^{-1}}]$"
        set ylabel "$T$"
        
        plot \
            dataV u 1:(1+$2) w l lc 'grey' t 'vzduch', \
            \
            data1 using ($1 >= 3925 ? $1 : 1/0):2 w l lc 'red' t 'vzorek 1', \
            data1 using (3925 >= $1 && $1 >= 3560 ? $1 : 1/0):2 w l lc '#dd9999' notitle, \
            data1 using (3560 >= $1 && $1 >= 2387 ? $1 : 1/0):2 w l lc 'red' notitle, \
            data1 using (2387 >= $1 && $1 >= 2300 ? $1 : 1/0):2 w l lc '#dd9999' notitle, \
            data1 using (2300 >= $1 && $1 >= 1873 ? $1 : 1/0):2 w l lc 'red' notitle, \
            data1 using (1873 >= $1 && $1 >= 1534 ? $1 : 1/0):2 w l lc '#dd9999' notitle, \
            data1 using (1534 >= $1 && $1 >= 705  ? $1 : 1/0):2 w l lc 'red' notitle, \
            data1 using (705 >= $1 && $1 >= 660 ? $1 : 1/0):2 w l lc '#dd9999' notitle, \
            data1 using (660 >= $1 ? $1 : 1/0):2 w l lc 'red' notitle, \
            \
            data2 using ($1 >= 3925 ? $1 : 1/0):2 w l lc 'green' t 'vzorek 1', \
            data2 using (3925 >= $1 && $1 >= 3560 ? $1 : 1/0):2 w l lc '#99dd99' notitle, \
            data2 using (3560 >= $1 && $1 >= 2387 ? $1 : 1/0):2 w l lc 'green' notitle, \
            data2 using (2387 >= $1 && $1 >= 2300 ? $1 : 1/0):2 w l lc '#99dd99' notitle, \
            data2 using (2300 >= $1 && $1 >= 1873 ? $1 : 1/0):2 w l lc 'green' notitle, \
            data2 using (1873 >= $1 && $1 >= 1503 ? $1 : 1/0):2 w l lc '#99dd99' notitle, \
            data2 using (1503 >= $1 && $1 >= 690  ? $1 : 1/0):2 w l lc 'green' notitle, \
            data2 using (690 >= $1 && $1 >= 660 ? $1 : 1/0):2 w l lc '#99dd99' notitle, \
            data2 using (660 >= $1 ? $1 : 1/0):2 w l lc 'green' notitle, \
            \
            data3 using ($1 >= 3925 ? $1 : 1/0):2 w l lc 'blue' t 'vzorek 1', \
            data3 using (3925 >= $1 && $1 >= 3560 ? $1 : 1/0):2 w l lc '#9999dd' notitle, \
            data3 using (3560 >= $1 && $1 >= 2387 ? $1 : 1/0):2 w l lc 'blue' notitle, \
            data3 using (2387 >= $1 && $1 >= 2300 ? $1 : 1/0):2 w l lc '#9999dd' notitle, \
            data3 using (2300 >= $1 && $1 >= 1873 ? $1 : 1/0):2 w l lc 'blue' notitle, \
            data3 using (1873 >= $1 && $1 >= 1490 ? $1 : 1/0):2 w l lc '#9999dd' notitle, \
            data3 using (1490 >= $1 && $1 >= 1422 ? $1 : 1/0):2 w l lc 'blue' notitle, \
            data3 using (1422 >= $1 && $1 >= 1384 ? $1 : 1/0):2 w l lc '#9999dd' notitle, \
            data3 using (1384 >= $1 && $1 >= 675  ? $1 : 1/0):2 w l lc 'blue' notitle, \
            data3 using (675 >= $1 && $1 >= 660 ? $1 : 1/0):2 w l lc '#9999dd' notitle, \
            data3 using (660 >= $1 ? $1 : 1/0):2 w l lc 'blue' notitle

    \end{gnuplot}
    \caption{Měření transmitance polymerových folií. Rysy způsobené fluktuací složení vzduchu jsou vykresleny světlejší a méně sytou barvou.}
    \label{pol-trans}
\end{figure}

% \begin{figure}[p]
%     \centering
%     \begin{gnuplot}[terminal=epslatex,terminaloptions={color size 18cm, 8cm}]
        
%         dataT = 'data/SKLO_GRNO.PRN'
%         dataR = 'data/SKLO_REFL_GRNO.PRN'

%         set ylabel 'T'
%         set y2label 'R'

%         plot dataT w l notitle, dataR w l notitle axes x1y2

%     \end{gnuplot}
%     \caption{Měření transmitance a odrazivosti skla}
%     \label{sklo-trans-refl}
% \end{figure}


\pagebreak


\section{Diskuse}
Bylo to špatně protože \eqref{A} 


\section{Závěr}
Bylo to hezké. assadfasd

\section{Literatura}
\bibliography{literatura} 
\bibliographystyle{ieeetr}

 
\end{document}