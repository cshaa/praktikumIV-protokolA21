% !TEX program = xelatex

% Nejlepší zážitek zaručí:
%
% TeX distribuce: texlive-full
%
% Editor:
%   VS Code s doplňky
%       * LaTeX Workshop
%       * LaTeX Utilities
%       * Gnuplot
%
% Další závislosti:
%   latexmk
%   bibtex
%   gnuplot


% Jak používat:
% Zkompilovat: make
% Gnuplot: make gnuplot
% Vyčistit: make clean


% Základní balíčky
\documentclass[10pt,a4paper]{article}
\usepackage[utf8]{inputenc}
\usepackage[czech]{babel}
\usepackage{graphicx}
\usepackage{lmodern}
\usepackage{amsmath}
\usepackage{hyperref}
\usepackage{gensymb}
\usepackage[top = 1.5cm, bottom = 1.5cm, left = 1.5cm, right = 1.5cm]{geometry}

% Bibtex
\usepackage{etoolbox}
\patchcmd{\thebibliography}{\section*{\refname}}{}{}{}

% Pro titulní stránku
\usepackage{titlesec}
\usepackage{setspace}
\usepackage{framed}
\usepackage{array}

% Vlastní balíčky 
\usepackage{gnuplottex}
\usepackage{epstopdf}
\usepackage{csvsimple}
\usepackage{units}


\renewcommand{\U}[1]{\ensuremath{\,\mathrm{#1}}}
\newcommand{\°}{\degree}

\newcommand{\titjmeno}{Michal Grňo}
\newcommand{\titobor}{FOF}


\newcommand{\titcislo}{A20}
\newcommand{\titnazev}{Fourierovská infračervená spektroskopie}
\newcommand{\titmereni}{4. 11. 2019}
\newcommand{\titodevzdani}{17. 11. 2019}


\begin{document}


\thispagestyle{empty}
\newgeometry{top = 2.5cm, bottom = 0cm, left = 2.5cm, right = 3cm}

{%T tomto je uzavřena celá titulka
%Tloušťka rámečku
\setlength{\fboxrule}{1.5pt}

\noindent
\framebox{
\begin{minipage}{\textwidth}
\setlength{\parindent}{17.62482 pt}
\phantom{d}

\begin{minipage}{0.6\textwidth}
{
\Large Kabinet výuky obecné fyziky, UK MFF\\
}
\vspace*{0.2cm}

{
\bfseries
\huge Fyzikální praktikum %ČÍSLO
}
\end{minipage}
\begin{minipage}{0.4\textwidth}
\begin{center}
\includegraphics[width=4.5cm]{ZFP.jpg}
\end{center}
\end{minipage}\\\\

%\vspace*{0.5cm}

{
\setstretch{1.5}
\Large
\noindent
Úloha č. \titcislo

\noindent
Název úlohy: \titnazev

\noindent
Jméno: \titjmeno
\hspace*{\fill}
Obor: \titobor

\noindent
Datum měření: \titmereni
\hspace*{\fill}
Datum odevzdání: \titodevzdani

\phantom{d}
}
\end{minipage}
}
%Konec horního rámečku

{
\phantom{d}

\Large
Připomínky opravujícího:\\
\vspace*{6.75cm}
}

\newcommand{\linka}{\noalign{\hrule height 1pt}}
\newcommand{\linkadva}{\noalign{\hrule height 1.5pt}}
\setlength\extrarowheight{9.5pt}
\Large
\noindent
\begin{tabular}{!{\vrule width 1.5pt} l !{\vrule width 1pt} c !{\vrule width 1pt} c !{\vrule width 1.5pt}}
\linkadva
   & Možný počet bodů & Udělený počet bodů \\\linkadva
  Práce při měření & 0-3 &  \\\linka
  Teoretická část & 0-2 &  \\\linka
  Výsledky a zpracování měření & 0-9 &  \\\linka
  Diskuse výsledků & 0-4 &  \\\linka
  Závěr & 0-1 &  \\\linka
  Použitá literatura & 0-1 &  \\\linkadva
  \hspace*{\fill} \textbf{Celkem} \hspace*{\fill}& max. 20 &  \\
\linkadva
\end{tabular}
\phantom{d}

Posuzoval: \hspace*{\fill}dne:~~~~~~~~~~~~~~~~~

}%Konec uzavření titulky
\newpage
\newgeometry{top = 2cm, bottom = 2cm, left = 2cm, right = 2cm}
\setcounter{page}{1}

\section{Pracovní úkoly}
\begin{enumerate}
    \item Proměřte rotačně-vibrační absorpční spektrum oxidu uhelnatého ve spektrální oblasti 2000 – 2500 $\U{cm^{-1}}$. Polohy absorpčních pásů zpracujte graficky a lineární regresí určete parametry vystupující v modelu pružného rotátoru pro základní vibrační stav molekuly a první excitovaný vibrační stav. Z těchto parametrů určete vzdálenosti jader uhlíku a kyslíku v základním a prvním excitovaném vibračním stavu.
    \item Spočtěte teplotní a tlakové rozšíření absorpčních pásů, určete rozdíl vibrační frekvence pro isotopomery $^{12}$C$^{16}$O a $^{13}$C$^{16}$O. Porovnejte tyto hodnoty s rozlišením použitého spektrálního přístroje.
    \item Změřte spektrum bez vzorku, určete oblasti absorpce oxidu uhličitého a vodních par v optické dráze spektrometru. Interpretujte nejvýraznější pásy absorpce CO$_2$.
    \item Proměřte spektra propustnosti polyetylénové a polypropylénové folie a interpretujte nejvýraznější pásy.
    \item Proměřte spektra propustnosti a odrazivosti skleněné a safírové destičky. Diskutujte rozdíl mezi oběma vzorky.
\end{enumerate}


\section{Teoretická část}

\subsection{Úvod}

Infračervená spektroskopie je nástroj používaný k identifikaci molekul přítomných v látce podle módů jejich mechanických kmitů. Konkrétně Fourierovská infračervená spektroskopie detekuje změny ve spektru průchozího elektromagnetického záření způsobené interakcí s kmitající molekulou.

U jednoduchých molekul lze v jejich spektru pozorovat tři charakteristické oblasti –
nízkofrekvenční\footnote{Nízkofrekvenční pás P bude v nižších hodnotách $f \propto \tilde{\nu}$, v grafu to ovšem znamená, že bude vpravo\dots} pás P odpovídající oblasti, kde se rotační energie a energie fotonu skládají, střední pás Q odpovídající vibračním přechodům beze změny rotačního stavu (a kvantového čísla $J$), vysokofrekvenční\footnote{\dots pás R bude naopak mít vysoké hodnoty $\tilde\nu$ a v grafu bude vlevo.} oblast R, kde energie fotonů přispívá jak vibračním, tak rotačním stavům. \cite{studijni-text}

\subsection{Energie pásů}

Z vlnočtu pásů, který naměříme pomocí spektroskopie, lze snadno vypočítat jejich energii:
\begin{equation}
    E = \mathrm{h} \mathrm{c} \tilde\nu,
    \label{energie}
\end{equation}
kde $\mathrm{h}$ je Planckova konstanta, $\mathrm{c}$ rychlost světla a $\tilde\nu = \lambda^{-1}$ vlnové číslo. U dvouatomové molekuly pro energii odpovídající pásům $P$ a $R$ při stavu $J$ v excitovaném stavu platí vztah:
\begin{equation}
    \frac{ R_J  - P_J }{ 2J + 1 } \; \frac{1}{\mathrm{h}}
    =  \left( 2B_1 - 3D_1 \right)  - D_1  \left( 2J+1 \right)^2,
    \label{regrese-excitovany}
\end{equation}
a pro molekuly v základním stavu:
\begin{equation}
    \frac{ R_{J-1}  - P_{J+1} }{ 2J + 1 } \; \frac{1}{\mathrm{h}}
    =  \left( 2B_0 - 3D_0 \right)  - D_0  \left( 2J+1 \right)^2,
    \label{regrese-zakladni}
\end{equation}
kde $B_0, B_1, D_0, D_1$ jsou konstanty charakteristické pro danou molekulu \cite{studijni-text}. Pokud známe závislosti $R_J(J)$ a $P_J(J)$, tyto konstanty lze určit lineární regresí z rovnic \eqref{regrese-excitovany} a \eqref{regrese-zakladni}.

\subsection{Vzdálenost jader}

Z \cite{studijni-text} víme, že pro dvouatomové molekuly platí $I=\mu r^2$ a $B=h / \left( 8 \pi^2 I \right)$, kde $\mu = m_1 m_2 / \left( m_1 + m_2 \right)$ je redukovaná hmotnost molekuly, $I$ je moment setrvačnosti molekuly, $r$ je vzdálenost jader atomů a $m_1, m_2$ jsou hmotnosti jejich jader. Z těchto rovnic si můžeme vyjádřit $r$ jako funkci $m_1, m_2$ a $B$:
\begin{equation}
    r = \sqrt{ \frac{1}{B} \; \frac{m_1 + m_2}{m_1 m_2} \; \frac{h}{8\pi^2} }.
    \label{vzdalenost-atomu}
\end{equation}
Předpokládáme, že hmotnosti atomů je možné z tabulek vyčíst s libovolnou přesností, proto je chyba $r$ určena pouze chybou $B$:
\begin{equation}
    \Delta r = \left| \frac{\partial r}{\partial B} \; \Delta B \right| = r \; \frac{1}{2} \; \frac{\Delta B}{B}.
    \label{vzdalenost-atomu-chyba}
\end{equation}
Protože $B$ známe pro základní i excitovaný stav z rovnic \eqref{regrese-excitovany} a \eqref{regrese-zakladni}, můžeme použít rovnici \eqref{vzdalenost-atomu} pro výpočet vzdálenosti jader atomu při základním a excitovaném stavu.


\subsection{Rozlišení (teplotní a tlakové)}
Je-li elektronový systém molekul v základním stavu (což je typické pro CO a CO$_2$ za obvyklých podmínek), přirozené šířky pásů jsou velice malé a pozorovaná šířka je téměř výhradně způsobená pohybem molekuly a jejími srážkami s okolními molekulami. \cite{studijni-text}

První ze zmíněných jevů – rozšíření pásu vlivem pohybu molekuly – je způsoben Dopplerovým posunem procházejícího paprsku vzhledem k letící molekule a závisí výhradně na teplotě vzorku. Rozšířený pás má Lorentzovský profil a pro jeho FWHM\footnote{FWHM = plná šířka v polovině maxima} platí vztah:
\begin{equation}
    \delta\tilde\nu_T = \frac{\tilde\nu_0}{\mathrm{c}} \sqrt{\frac{8\mathrm{k}T}{m} \ln 2},
    \label{fwhm-teplota}
\end{equation}
kde $\tilde\nu_0$ je vlnočet maxima, $\mathrm{c}$ rychlost světla, $\mathrm{k}$ Boltzmannova konstanta, $m$ hmotnost molekuly a $T$ je teplota vzorku. \cite{studijni-text}

Druhým jevem, který má vliv na rozšíření pásu, jsou srážky s okolními molekulami. Toto rozšíření je pro \textit{čistý jednomolekulový plyn} závislé výhradně na tlaku, má opět Lorentzovský profil a pro jeho FWHM platí vztah\footnote{Vztah buď vyčíst v tabulkách, nebo vyvodit z obecného vztahu 2.25 v \cite{studijni-text} po dosazení odpovídajících hodnot, tedy $T_{REF} = T$ a $p_s = p$}:
\begin{equation}
    \delta\tilde\nu_p = 2 \gamma p,
    \label{fwhm-tlak}
\end{equation}
kde $p$ je tlak plynu a $\gamma$ je pološířka rozšíření (HWHM) na jednotku tlaku charakteristická pro daný plyn – tu je možné vyčíst v tabulkách.

Protože konvolucí dvou Lorentzových funkcí o pološířkách $\ell_1$ a $\ell_2$ dostaneme Lorenzovu funkci o pološířce $\ell = \ell_1 + \ell_2$ \cite{libretexts:lorentzian-convolution}, bude pro šířku pásu platit
\begin{equation}
    \delta\tilde\nu = \delta\tilde\nu_T + \delta\tilde\nu_p.
    \label{fwhm-total}
\end{equation}

V experimentu chceme určit, zda je možné odlišit od sebe dva izotopology\footnote{Izotopology jsou molekuly, které jsou totožné až na izotopové záměny.} $^{12}$C$^{16}$O a $^{13}$C$^{16}$O. Z \cite{studijni-text} víme, že pro úhlovou frekvenci platí $\omega = \sqrt{f/\mu}$, tedy $\tilde\nu \propto \mu^{-\nicefrac{1}{2}}$. Známe-li tedy vlnočet nějakého pásu $\tilde\nu_A$ pro první z izomerů a redukované hmotnosti obou izomerů $\mu_A, \mu_B$, můžeme snadno spočítat vlnočet téhož pásu pro druhý z izomerů $\tilde\nu_B = \tilde\nu_A \sqrt{\mu_A / \mu_B}$. Rozdíl mezi vlnočty izomerů tedy bude:
\begin{equation}
    \Delta\tilde\nu = |\tilde\nu_A - \tilde\nu_B| = \tilde\nu_A \left|1 - \sqrt{\frac{\mu_A}{\mu_B}}\right|
    \label{rozdil-izotopu}
\end{equation}

Pásy izomerů považujeme za od sebe rozlišitelné, pokud se vlnočty jejich maxim od sebe liší více než o $\delta\tilde\nu$, tedy $\Delta\tilde\nu > \delta\tilde\nu$. Po dosazení z rovnic \eqref{fwhm-teplota} a \eqref{fwhm-tlak} a drobné úpravě dostáváme nerovnost, která je naší podmínkou rozlišitelnosti izotopologů:
\begin{equation}
    \left|1 - \sqrt{\frac{\mu_A}{\mu_B}}\right| \;\; > \;\; \sqrt{\frac{8\mathrm{k}T}{m \mathrm{c}^2} \ln 2} \; + \; \frac{2 \gamma p}{\tilde\nu}.
    \label{podminka-rozlisitelnosti}
\end{equation}


\subsection{Tabulkové hodnoty}

Hmotnosti významných izotopů uhlíku a kyslíku jsme převzali z \cite{isotopic-masses}:
\begin{table}[h!]
    \centering
    \begin{tabular}{ c|r }
        \bfseries izotop & \multicolumn{1}{c}{$m \; [\U{u}]$}
        \csvreader[ head to column names ]{co_m.csv.tmp}{}
        {
            \csviffirstrow{\\\hline}{\\}
            \colqty & \colvalue
        }
    \end{tabular}
    \caption{Hmotnosti izotopů C a O}
    \label{hmotnosti}
\end{table}

Z těchto je hned možné vypočítat redukované hmotnosti izotopologů oxidu uhelnatého:
\begin{table}[h!]
    \centering
    \begin{tabular}{ c|r }
        \bfseries izotopolog & \multicolumn{1}{c}{$\mu \; [\U{u}]$}
        \csvreader[ head to column names ]{co_mu.csv.tmp}{}
        {
            \csviffirstrow{\\\hline}{\\}
            \colqty & \colvalue
        }
    \end{tabular}
    \caption{Redukované hmotnosti izotopologů CO}
    \label{redukovane-hmotnosti}
\end{table}

Pro převod z atomových jednotek na jednotky SI platí následující vztah:
\begin{equation}
    1 \U{u} = 1.6605390666050 \cdot 10^{-27} \U{kg}
    \label{atomic-mass}
\end{equation}

\subsection{Chyba nepřímého měření}

\textbf{Pozn.:} Všude, kde není uvedeno jinak, využíváme toho, že jsou chyby malé a pro chybu nepřímého měření tedy~platí:
\begin{align}
    f &= f(x_1, x_2, \dots) &
    \Delta f &= \sqrt{ \sum_i \left(
        \tfrac{\partial f}{\partial x_i} \; \Delta x_i
    \right)^2 }
    \label{chyba}
\end{align}

\section{Výsledky měření}

\subsection{Oxid uhelnatý (CO)}


\begin{figure}[p]
    \centering
    \begin{gnuplot}[terminal=epslatex,terminaloptions={color size 18cm, 8cm}]
        data = 'data/CO_GRNO.PRN'
        set xrange [2500:2000] reverse
        set xlabel "$\\tilde\\nu \\; [\\U{cm^{-1}}]$"
        set ylabel "$T$"
        plot \
            data using ($1 >= 2385 ? $1 : 1/0):2 w l lc black notitle, \
            data using (2385 >= $1 && $1 >= 2349 ? $1 : 1/0):2 w l t "$CO_2, R$", \
            data using (2349 >= $1 && $1 >= 2290 ? $1 : 1/0):2 w l t "$CO_2, P$", \
            data using (2290 >= $1 && $1 >= 2238 ? $1 : 1/0):2 w l lc black notitle, \
            data using (2238 >= $1 && $1 >= 2143 ? $1 : 1/0):2 w l t "$CO, R$", \
            data using (2143 >= $1 && $1 >= 2033 ? $1 : 1/0):2 w l t "$CO, P$", \
            data using (2033 >= $1 ? $1 : 1/0):2 w l lc black notitle
    \end{gnuplot}
    \caption{Měření transmitance kyvety s oxidem uhelnatým (CO). Postupně vidíme (zleva) pásy R a P oxidu uhličitého (CO$_2$), pásy R a P oxidu uhelnatého (CO). Pás Q pozorovatelný nebyl.}
    \label{co-trans}
\end{figure}

\begin{figure}[p]
    \centering
    \begin{gnuplot}[terminal=epslatex,terminaloptions={color size 15cm,10cm}]

        load 'matlib.cfg'
        h = 6.62607015e-34
        c = 299792458
        m_u = 1.6605390666050e-27
        k_B = 1.380649e-23

        LF_File = 'data/co_tab_grno.txt'
        LF_Columns = 6
        load 'loadfile.cfg'
        N = LF_Rows - 1

        stred_i = -1

        array Energie[N]
        do for [i = 1:N] {
            Energie[i] = real(LF_Col1[i+1]) * 100 * h * c
            if (stred_i == -1 && Energie[i] >= 214500*h*c) {
                stred_i = i
            }
        }

        array P[stred_i - 1 + 1]
        array R[N - stred_i]
        do for [i = 1:N] {
            if (i <= stred_i) { P[stred_i - i + 1] = Energie[i] }
            else { R[i - stred_i] = Energie[i] }
        }

        set multiplot layout 2,1

        set yrange [4:4.5]
        set ytics 0.1
        set xlabel "$J$"
        set ylabel "$P_J, R_J \\; [10^{-20} \\U{J}]$"
        set key outside
        set key right center

        plot sample \
            [i=1:|P|:1] '+' using (i):(P[i]*1e20) pt 5 t "$P_J$", \
            [i=1:|R|:1] '+' using (i-1):(R[i]*1e20) pt 4 t "$R_J$"


        YSCALE = 1 / 1.e11;

        N = min(|R|+1, |P|)

        array X1[N]
        array Y1[N]
        array Y1_scaled[N]

        set print 'fit1.dat.tmp'
        f1(x) = a1 * x + b1

        do for [i=2:N] {
            X1[i] = (2*i-1)**2
            Y1[i] = (R[i-1] - P[i]) / ( (2*i-1) * h)
            Y1_scaled[i] = Y1[i] * YSCALE

            print sprintf("%f %f", X1[i], Y1[i])
        }

        fit f1(x) 'fit1.dat.tmp' via a1, b1
        f1_scaled(x) = f1(x) * YSCALE


        M = min(|R|+2, |P|-1)

        array X2[M]
        array Y2[M]
        array Y2_scaled[M]

        set print "fit2.dat.tmp"
        f2(x) = a2 * x + b2

        do for [i=3:M] {
            X2[i] = (2*i-1)**2
            Y2[i] = (R[i-2] - P[i+1]) / ( (2*i-1) * h)
            Y2_scaled[i] = Y2[i] * YSCALE

            print sprintf("%f %f", X2[i], Y2[i])
        }

        fit f2(x) 'fit2.dat.tmp' via a2, b2
        f2_scaled(x) = f2(x) * YSCALE


        set xtics rotate
        set xrange [0:1800]
        set yrange [1.13:1.16]
        set ytics   0.01
        set xlabel "$\\left(2J + 1\\right)^2$"
        set ylabel "$\\left(P_a - R_b\\right)/\\mathrm{h}\\left(2J + 1\\right)$\n$[10^{11}\\U{s^{-1}}]$" font ",40"
        plot sample \
            [i=2:N:1] '+' using (X1[i]):(Y1_scaled[i]) pt 5 lc 1 notitle, \
            [i=3:M:1] '+' using (X2[i]):(Y2_scaled[i]) pt 4 lc 2 notitle, \
            f1_scaled(x) lc 1 notitle, f2_scaled(x) lc 2 notitle, \
            '+' using (1/0) w lp pt 5 lc 1 t "$R_J - P_J$", \
            '+' using (1/0) w lp pt 4 lc 2 t "$R_{J-1} - P_{J+1}$"



        b1_exp = exponent(b1)
        b2_exp = exponent(b1)
        a1_exp = exponent(a1)
        a2_exp = exponent(a2)

        set print "fitCO.csv.tmp"
        print sprintf("colqty,colunit,colvalue,colstdev")
        print sprintf("$2B_1 - 3D_1$,$[10^{%d} \\U{s^{-1}}]$,%.5f,%.5f", b1_exp, manexp(b1,-b1_exp), manexp(b1_err,-b1_exp))
        print sprintf("$2B_0 - 3D_0$,$[10^{%d} \\U{s^{-1}}]$,%.5f,%.5f", b2_exp, manexp(b2,-b2_exp), manexp(b2_err,-b2_exp))
        print sprintf("$D_1$,$[10^{%d} \\U{s^{-1}}]$,%.2f,%.2f", a1_exp, manexp(-a1, -a1_exp), manexp(a1_err, -a1_exp))
        print sprintf("$D_0$,$[10^{%d} \\U{s^{-1}}]$,%.2f,%.2f", a2_exp, manexp(-a2, -a2_exp), manexp(a2_err, -a2_exp))


        B1 = .5*(b1 - 3*a1)
        B1_err = .5*sqrt( b1_err**2 - (3*a1_err)**2 )

        B0 = .5*(b2 - 3*a2)
        B0_err = .5*sqrt( b2_err**2 - (3*a1_err)**2 )
        
        B0_exp = exponent(B0)
        B1_exp = exponent(B1)

        print sprintf("$B_1$,$[10^{%d} \\U{s^{-1}}]$,%.4f,%.4f", B1_exp, manexp(B1, -B1_exp), manexp(B1_err, -B1_exp))
        print sprintf("$B_0$,$[10^{%d} \\U{s^{-1}}]$,%.4f,%.4f", B0_exp, manexp(B0, -B1_exp), manexp(B0_err, -B0_exp))


        mC12 = 12.000000
        mC13 = 13.003355
        mO16 = 15.994915
        mO17 = 16.999132
        set print "co_m.csv.tmp"
        print "colqty,colvalue"
        print sprintf("$^{12}$C,%.6f",mC12)
        print sprintf("$^{13}$C,%.6f",mC13)
        print sprintf("$^{16}$O,%.6f",mO16)
        print sprintf("$^{17}$O,%.6f",mO17)

        mu12 = (mC12 * mO16)/(mC12 + mO16)
        mu13 = (mC13 * mO16)/(mC13 + mO16)
        set print "co_mu.csv.tmp"
        print "colqty,colvalue"
        print sprintf("$^{12}$C$^{16}$O,%.6f",mu12)
        print sprintf("$^{13}$C$^{16}$O,%.6f",mu13)

        konst = h/(8*pi**2)
        ang = 1e10 # 1e10 angstrem = 1 m
        r12_0 = sqrt( (1./B0) * konst / (mu12 * m_u) ) * ang
        r12_1 = sqrt( (1./B1) * konst / (mu12 * m_u) ) * ang
        r13_0 = sqrt( (1./B0) * konst / (mu13 * m_u) ) * ang
        r13_1 = sqrt( (1./B1) * konst / (mu13 * m_u) ) * ang
        r12_0_err = .5 * r12_0 * B0_err / B0
        r12_1_err = .5 * r12_1 * B1_err / B1
        r13_0_err = .5 * r13_0 * B0_err / B0
        r13_1_err = .5 * r13_1 * B1_err / B1
        set print "co_r.csv.tmp"
        print "colqty,colvalue,colstdev"
        print sprintf("$r_{12,0}$,%.5f,%.5f", r12_0, r12_0_err)
        print sprintf("$r_{12,1}$,%.5f,%.5f", r12_1, r12_1_err)
        print sprintf("$r_{13,0}$,%.5f,%.5f", r13_0, r13_0_err)
        print sprintf("$r_{13,1}$,%.5f,%.5f", r13_1, r13_1_err)

        delta_nu_T = 2130*sqrt( (300*8*k_B*log(2))/(c**2 * (mC12+mO16) * m_u) )
        set print "delta_nu_t.csv.tmp"
        print "colvalue"
        print sprintf("%.3f",delta_nu_T)

    \end{gnuplot}
    \caption{Energie pásů P a R v CO. V horním grafu jsou vyznačeny energie spektrálních čar CO (viz graf v obrázku č. \ref{co-trans}) v závislosti na jejich kvantovém čísle $J$. V obou pásech roste $J$ směrem od středu pozorovaného obrazce ven, v pásu P má první absorpční maximum $J=1$, v pásu R má první maximum $J=0$. Ve spodním grafu potom vidíme závislost z rovnic \eqref{regrese-excitovany} a \eqref{regrese-zakladni} proloženou přímkou.}
    \label{co-fit}
\end{figure}

Jako první vzorek jsme měřili transmitanci kyvety se stěnami z CaF$_2$ obsahující CO pod tlakem $7 \U{mbar}$. Měření bylo předem kalibrováno na transmitanci vzduchu a probíhalo na rozsahu $2000$ až $2500 \U{cm^{-1}}$ při rozlišení $0.35 \U{cm^{-1}}$. Protože se mezi kalibrací a měřením změnil podíl CO$_2$ v měřeném prostoru, bylo kromě spektra CO viditelné i spektrum CO$_2$. V obou případech byla jasně rozlišitelná pásma P a R, pásmo Q nebylo pozorováno. Zatímco v případě CO jsou jasně rozlišitelná absorpční maxima\footnote{Maxima absorpce, tj. minima naměřené intenzity.} odpovídající pásům $P_J$ a $R_J$ pro celočíselné hodnoty $J$, spektrum CO$_2$ vykazuje určitou míru překryvu a jeho maxima nejsou jasně oddělená. Naměřené spektrum bylo zaneseno do grafu na obrázku č. \ref{co-trans}.

Naměřená absorpční maxima jsme potom odstředu očíslovali a jejich vlnočet jsme převedli na energii podle vzorce \eqref{energie}. Pás P začíná od jedničky $P_1 = 4.288 \cdot 10^{-20} \U{J}$ a pokračuje k vyšším hodnotám (v grafu doleva), pás R začíná od nuly $R_0 = 4.280 \cdot 10^{-20} \U{J}$ a pokračuje k nižším hodnotám (v grafu doprava). Takto vypočtené a očíslované energie $P_J$ a $R_J$ jsme v závislosti na jejich kvantovém čísle $J$ vynesli do grafu na obrázku č. \ref{co-fit} nahoře.

Následně jsme podle vzorců \eqref{regrese-excitovany} a \eqref{regrese-zakladni} vynesli závislost $\left(P_a - R_b\right)/\mathrm{h}\left(2J + 1\right)$ na $\left(2J + 1\right)^2$ do grafu na obrázku č. \ref{co-fit} dole. Pomocí lineární regrese metodou nejmenších čtverců jsme nalezli konstanty $B_0, B_1, D_0, D_1$.

\begin{table}[h!]
    \centering
    \begin{tabular}{cc|rl}
        \multicolumn{2}{c}{\bfseries veličina \& jednotky} &
        \multicolumn{2}{c}{\bfseries hodnota}
        \csvreader[ head to column names ]{fitCO.csv.tmp}{}
        {
            \csviffirstrow{\\\hline}{\\}
            \colqty & \colunit & \colvalue & $\pm$ \colstdev
        }
    \end{tabular}
    \caption{Konstanty charakteristické pro molekulu CO získané regresí. První čtyři řádky odpovídají přímo parametrům fitu dle rovnic \eqref{regrese-excitovany} a \eqref{regrese-zakladni}, poslední dva řádky jsme dostali prostým vyjádřením $B$.}
    \label{tab-co}
\end{table}

Poté jsme použili známé redukované hmotnosti z tabulky č. \ref{redukovane-hmotnosti} a pomocí vzorce \eqref{vzdalenost-atomu} jsme vypočetli vzdálenost jader v molekule CO. Protože v grafu na obrázku č. \ref{co-trans} nebyl pozorován žádný izotopolog, bez dalších informací nebylo jednoznačné, o který izotop uhlíku se jedná. Výpočet $r$ jsme tudíž provedli pro $^12$C$^16$O, i pro $^13$C$^16$O. Protože je ale izotop $^12$C mnohem běžnější, velice pravděpodobně jsou správné hodnoty $r_{12,0}$ a $r_{12,1}$.

\begin{table}[h!]
    \centering
    \begin{tabular}{ r|rl }
        \bfseries vel. &
        \multicolumn{2}{c}{ $[10^{-10} \U{m}]$ }
        \csvreader[ head to column names ]{co_r.csv.tmp}{}
        {
            \csviffirstrow{\\\hline}{\\}
            \colqty & \colvalue & $\pm$ \colstdev
        }
    \end{tabular}
    \caption{Vzdálenosti jader izotopologů CO v základním a excitovaném stavu.}
    \label{tab-co-vzdalenosti}
\end{table}

Izotopology $^12$C$^16$O a $^13$C$^16$O jsme nepozorovali, bylo by tedy vhodné ověřit, zda jsou vůbec za našich podmínek rozlišitelné. Pásy CO jsme pozorovali mezi $\tilde\nu = 2040$ a $2230$. Podle \eqref{rozdil-izotopu} vypočteme, že maxima izotopologů by od sebe měla být vzdálena:
\begin{equation}
    \Delta\tilde\nu \approx 2130 \U{cm^{-1}} \;
    \left| 1 - \sqrt{\frac{\mu(^{12}C^{16}O)}{\mu(^{13}C^{16}O)}} \right|
    \approx 2130 \cdot 0.022 \U{cm^{-1}} \approx 50 \U{cm^{-1}}.
\end{equation}
To je o několik řádů vyšší než rozlišení přístroje, které je $0.35 \U{cm^{-1}}$. Rozlišení přístroje tedy pozorování izotopologů nebrání.

Vypočteme ještě teplotní a tlakové „rozostření“. Podle vzorce \eqref{fwhm-teplota} platí:
\begin{equation}
    \delta\tilde\nu_T \approx
    (2130 \U{cm^{-1}}) \; \sqrt{300 \U{K}} \; \sqrt{\frac{8 \mathrm{k}}{m_{CO} \; c^2} \ln 2} \approx
    \csvreader[ head to column names ]{delta_nu_t.csv.tmp}{}{\colvalue} \U{cm^{-1}}.
\end{equation}
Podle vzorce \eqref{fwhm-tlak}, kde $\gamma \approx 0.05 \U{cm^{-1} / bar}$ \cite{studijni-text} a $p = 7 \U{mbar}$ platí:
\begin{equation}
    \delta\tilde\nu_p \approx 0.05 \cdot 0.007 \U{cm^{-1}} \approx 4 \cdot 10^{-5} \U{cm^{-1}}.
\end{equation}
Je tedy zřejmé, že $\delta\tilde\nu$ je řádově menší než $\Delta\tilde\nu$ a podmínka rozlišitelnosti izomerů \eqref{podminka-rozlisitelnosti} je splněna.



\subsection{Vzduch a polymerové folie}

\begin{figure}[p]
    \centering
    \begin{gnuplot}[terminal=epslatex,terminaloptions={color size 18cm, 8cm}]
        data = 'data/VZDUCH_GRNO.PRN'
        set key left top
        set xrange [4500:500] reverse
        set xlabel "$\\tilde\\nu \\; [\\U{cm^{-1}}]$"
        set ylabel "$T$"
        plot \
            data using ($1 >= 4000 ? $1 : 1/0):2 w l lc black notitle, \
            data using (4000 >= $1 && $1 >= 3500 ? $1 : 1/0):2 w l t "$H_2O$, 1. h." lc 'green', \
            data using (3500 >= $1 && $1 >= 2400 ? $1 : 1/0):2 w l lc black notitle, \
            data using (2400 >= $1 && $1 >= 2240 ? $1 : 1/0):2 w l t "$CO_2$, 1. h.", \
            data using (2240 >= $1 && $1 >= 2050 ? $1 : 1/0):2 w l lc black notitle, \
            data using (2050 >= $1 && $1 >= 1330 ? $1 : 1/0):2 w l t "$H_2O$, 2. h.", \
            data using (1330 >= $1 && $1 >= 705  ? $1 : 1/0):2 w l lc black notitle, \
            data using (705 >= $1 && $1 >= 670.5 ? $1 : 1/0):2 w l t "$CO_2, R$, 2. h." lc 'red', \
            data using (670.5 >= $1 && $1 >= 666 ? $1 : 1/0):2 w l t "$CO_2, Q$, 2. h.", \
            data using (666 >= $1 && $1 >= 620 ? $1 : 1/0):2 w l t "$CO_2, P$, 2. h.", \
            data using (620 >= $1 ? $1 : 1/0):2 w l lc black notitle
    \end{gnuplot}
    \caption{Měření transmitance vzduchu}
    \label{vzduch-trans}
\end{figure}

\begin{figure}[p]
    \centering
    \begin{gnuplot}[terminal=epslatex,terminaloptions={color size 18cm, 8cm}]

        data1 = 'data/POL1_GRNO.PRN'
        data2 = 'data/POL2_GRNO.PRN'
        data3 = 'data/POL3_GRNO.PRN'
        dataV = 'data/VZDUCH_GRNO.PRN'

        set key left bottom
        set xrange [4000:500] reverse
        set yrange [0:1.1]
        set xlabel "$\\tilde\\nu \\; [\\U{cm^{-1}}]$"
        set ylabel "$T$"
        
        plot \
            dataV u 1:(1+$2) w l lc 'grey' t 'vzduch', \
            \
            data1 using ($1 >= 3925 ? $1 : 1/0):2 w l lc 'red' t 'vzorek 1', \
            data1 using (3925 >= $1 && $1 >= 3560 ? $1 : 1/0):2 w l lc '#dd9999' notitle, \
            data1 using (3560 >= $1 && $1 >= 2387 ? $1 : 1/0):2 w l lc 'red' notitle, \
            data1 using (2387 >= $1 && $1 >= 2300 ? $1 : 1/0):2 w l lc '#dd9999' notitle, \
            data1 using (2300 >= $1 && $1 >= 1873 ? $1 : 1/0):2 w l lc 'red' notitle, \
            data1 using (1873 >= $1 && $1 >= 1534 ? $1 : 1/0):2 w l lc '#dd9999' notitle, \
            data1 using (1534 >= $1 && $1 >= 705  ? $1 : 1/0):2 w l lc 'red' notitle, \
            data1 using (705 >= $1 && $1 >= 660 ? $1 : 1/0):2 w l lc '#dd9999' notitle, \
            data1 using (660 >= $1 ? $1 : 1/0):2 w l lc 'red' notitle, \
            \
            data2 using ($1 >= 3925 ? $1 : 1/0):2 w l lc 'green' t 'vzorek 2', \
            data2 using (3925 >= $1 && $1 >= 3560 ? $1 : 1/0):2 w l lc '#99dd99' notitle, \
            data2 using (3560 >= $1 && $1 >= 2387 ? $1 : 1/0):2 w l lc 'green' notitle, \
            data2 using (2387 >= $1 && $1 >= 2300 ? $1 : 1/0):2 w l lc '#99dd99' notitle, \
            data2 using (2300 >= $1 && $1 >= 1873 ? $1 : 1/0):2 w l lc 'green' notitle, \
            data2 using (1873 >= $1 && $1 >= 1503 ? $1 : 1/0):2 w l lc '#99dd99' notitle, \
            data2 using (1503 >= $1 && $1 >= 690  ? $1 : 1/0):2 w l lc 'green' notitle, \
            data2 using (690 >= $1 && $1 >= 660 ? $1 : 1/0):2 w l lc '#99dd99' notitle, \
            data2 using (660 >= $1 ? $1 : 1/0):2 w l lc 'green' notitle, \
            \
            data3 using ($1 >= 3925 ? $1 : 1/0):2 w l lc 'blue' t 'vzorek 3', \
            data3 using (3925 >= $1 && $1 >= 3560 ? $1 : 1/0):2 w l lc '#9999dd' notitle, \
            data3 using (3560 >= $1 && $1 >= 2387 ? $1 : 1/0):2 w l lc 'blue' notitle, \
            data3 using (2387 >= $1 && $1 >= 2300 ? $1 : 1/0):2 w l lc '#9999dd' notitle, \
            data3 using (2300 >= $1 && $1 >= 1873 ? $1 : 1/0):2 w l lc 'blue' notitle, \
            data3 using (1873 >= $1 && $1 >= 1490 ? $1 : 1/0):2 w l lc '#9999dd' notitle, \
            data3 using (1490 >= $1 && $1 >= 1422 ? $1 : 1/0):2 w l lc 'blue' notitle, \
            data3 using (1422 >= $1 && $1 >= 1384 ? $1 : 1/0):2 w l lc '#9999dd' notitle, \
            data3 using (1384 >= $1 && $1 >= 675  ? $1 : 1/0):2 w l lc 'blue' notitle, \
            data3 using (675 >= $1 && $1 >= 660 ? $1 : 1/0):2 w l lc '#9999dd' notitle, \
            data3 using (660 >= $1 ? $1 : 1/0):2 w l lc 'blue' notitle

    \end{gnuplot}
    \caption{Měření transmitance polymerových folií. Rysy způsobené fluktuací složení vzduchu jsou vykresleny světlejší a méně sytou barvou.}
    \label{pol-trans}
\end{figure}

Dále jsme měřili bez vzorku a kalibrace transmitanci vzduchu od $\tilde\nu = 500$ do $4500 \U{cm^{-1}}$. Naměřené pektrum je v grafu na obrázku č. \ref{vzduch-trans}. Zajímavé oblasti byly: P-R pásy CO$_2$ na 2400; P-Q-R pásy CO$_2$ na 600; husté lesy čar mezi 3500--4000 a mezi 1300--2000 patřící vodním parám. Chaotická povaha čar naznačuje, že molekuly vody vibrují mnohem složitěji – například kvůli častým interakcím s ostatními molekulami H$_2$O pomocí vodíkových můstků.

Poté jsme již s kalibrací měřili transmitanci tří různých polymerových folií na rozsahu $\tilde\nu = 500$ až $4000 \U{cm^{-1}}$, jejich spektra jsou v grafu na obrázku č. \ref{pol-trans}. V grafu jsou jasně vidět rysy způsobené proměnlivostí složení vzduchu – obě oblasti odpovídající vodním parám i obě oblasti odpovídající CO$_2$ jsou pozorovatelné ve spektru folií. Všechny tři folie byly vyrobeny buďto z polyetylenu (PE), nebo z polypropylenu (PP), naším cílem je interpretovat významné rysy spekter jednotlivých vzorků a určit, který vzorek byl vyroben z jakého materiálu. Klíčová indície k rozlišení materiálů je, že PE je ztvořen dlouhými řetězci —CH$_2$—, zatímco PP má složitější řetězec —CH(CH$_3$)–CH$_2$—.

Prvním výrazným rysem je téměř úplná absorpce světla v oblasti 2840 až 2970 \U{cm^{-1}}, a to u všech tří vzorků. Zatímco vzorky 1 a 2 zablokovaly v této oblasti všechno světlo a lze u nich pozorovat pouze dlouhé plateau s $T=0$, u vzorku 3 vidíme, že se ve skutečnosti jedná o dvě absorpční maxima: první $\tilde\nu = 2920 \U{cm^{-1}}$ odpovídá antisymetrickým valenčním kmitům „$\nu_\mathrm{as}$“ skupiny —CH$_2$— a druhé $\tilde\nu = 2850 \U{cm^{-1}}$ odpovídá jejím symetrickým valenčním kmitům „$\nu_\mathrm{s}$“. \cite{studijni-text}. Skupina —CH$_2$— je společná polyetylenu i polypropylenu. PP má navíc skupinu —CH$_3$, jejíž valenční kmity by odpovídaly maximům na 2960 \U{cm^{-1}} a 2870 \U{cm^{-1}}. Vidíme, že vzorek 3 zřejmě taková maxima nemá, což je evidence pro to, že byl vyroben z PE. Druhé dva vzorky mají v této oblasti plateau, nemůžeme o nich tedy zatím říci nic.

Další oblastí, která stojí za zmínku, je maximum vzorku 1 na 2723 \U{cm^{-1}}. Toto maximum se za pomoci tabulek v \cite{studijni-text} nepodařilo vysvětlit.

Následuje oblast 1500 až 700 \U{cm^{-1}}, kde je výskyt maxim hojný. Nejprve je zde oblast 1475 až 1430 – vzorky 1 a 2 tu opět mají plateau, vzorek 3 dvojici peaků 1473 \U{cm^{-1}} a 1462 \U{cm^{-1}}. Tyto jsou nejblíže tabulkovým maximům 1470 pro nůžkovou deformaci „$\delta$“ skupiny —CH$_2$— a 1465 pro asymetrickou deformaci „$\delta_\mathrm{as}$“ skupiny —CH$_3$ \cite{studijni-text}. To by napovídalo, že vzorek 3 obsahuje i skupinu —CH$_3$, o vzorcích 1 a 2 stále nic říci nemůžeme.

Dále je zde maximum 1377, které sdílí vzorek 1 a 2. Toto maximum nejlépe odpovídá tabulkovému maximu $\tilde\nu = 1375 \U{cm^{-1}}$ pro symetrické deformace „$\delta_\mathrm{s}$“ skupiny —CH$_3$. To je evidence pro přítomnost —CH$_3$ skupiny ve vzorcích 1 a 2 a její absenci ve vzorku 3. Maximum vzorku 1 je vysoké 0.62, maximum vzorku 2 je vysoké 0.30. To lze vysvětlit tak, že vzorek 1 obsahuje větší množství —CH$_3$ skupin než vzorek 2.

V oblasti 1330 až 800 \U{cm^{-1}} má vzorek 1 les maxim, nejvýraznější z nich jsou 1168, 998, 973 a 841. Vzorek 3 zde má naopak téměř ploché spektrum. Vzorek 2 má v této oblasti spektrum "kopcovité", se vzorkem 1 sdílí pouze jedno maximum $\tilde\nu = 1303 \U{cm^{-1}}$ (kroutivá deformace „$\tau$“ skupiny –CH$_2$— \cite{studijni-text}), jinak se jeví nahodile. V tabulkách v \cite{studijni-text} se nepodařilo najít vysvětlení tohoto hojného výskytu maxim, pravděpodobně je ale výsledkem komplexních interakcí mezi několika různými funkčními skupinami – tj. něco, co bychom čekali u PP a nečekali u PE.

Poslední dvojici maxim sdílí vzorky 2 a 3. Jeden peak se nachází na $\tilde\nu = 730 \U{cm^{-1}}$, druhý na 720 \U{cm^{-1}}. Druhé maximum odpovídá kolébavé deformaci „$\varrho$“ skupiny —CO$_2$. Vzorek 1 má v této oblasti ploché spektrum.

Shrneme-li nyní pozorované rysy, dospějeme k následujícímu závěru: vzorky 1 a 3 jsou si nejméně podobné, vzorek 2 sdílí některé rysy se vzorkem 1 a některé se vzorkem 3. Vzorek 1 splňuje nejvíce kritérií, která by měl splňovat PP (jmenovitě maximum na $\tilde\nu = 1375 \U{cm^{-1}}$, les maxim mezi 1330 a 800 \U{cm^{-1}}) a nesplňuje některá kritéria, která by měl splňovat PE (maximum na $\tilde\nu = 720 \U{cm^{-1}}$). Oproti tomu vzorek 3 nesplňuje zmíněná kritéria pro PP a splňuje kritéria pro PE. S těmito informacemi můžeme učinit závěr, že \textbf{vzorek 1 byl z polypropylenu, vzorek 3 byl z polyetylenu a vzorek 2 byl z jejich směsi, anebo jiného podobného plastu}.

\begin{figure}[p]
    \centering
    \begin{gnuplot}[terminal=epslatex,terminaloptions={color size 18cm, 8cm}]
        
        dataT = 'data/SKLO_GRNO.PRN'
        dataR = 'data/SKLO_REFL_GRNO.PRN'

        set xrange [500:4000] reverse
        # set xlabel '$\\tilde\\nu [\\U{cm^{-1}}]$'
        set yrange [0:1]
        set y2range [0:1]
        set ylabel '$T$'
        set y2label '$R$'

        plot dataT w l notitle, dataR w l notitle axes x1y2

    \end{gnuplot}
    \caption{Měření transmitance a odrazivosti skla}
    \label{sklo-trans-refl}
\end{figure}

\begin{figure}[p]
    \centering
    \begin{gnuplot}[terminal=epslatex,terminaloptions={color size 18cm, 8cm}]
        
        dataT = 'data/SAFIR_GRNO.PRN'
        dataR = 'data/SAFIR_REFL_GRNO.PRN'

        set xrange [500:4000] reverse
        # set xlabel '$\\tilde\\nu [\\U{cm^{-1}}]$'
        set yrange [0:1]
        set y2range [0:1]
        set ylabel '$T$'
        set y2label '$R$'

        plot dataT w l notitle, dataR w l notitle axes x1y2

    \end{gnuplot}
    \caption{Měření transmitance a odrazivosti safíru}
    \label{safir-trans-refl}
\end{figure}


\pagebreak


\section{Diskuse}
Při výpočtu vzdálenosti atomů molekuly CO podle vzorce \eqref{vzdalenost-atomu} byla v redukované hmotnosti zanedbána vazebná energie molekuly, ta je řádově $10 \U{eV}$ \cite{wiki:carbon-monoxide}. Rychlý výpočet ukazuje, že $m_\mathrm{CO} \; c^2 \approx (12\U{u} + 16\U{u})c^2 \approx 10^{10} \U{eV}$, tedy vazbová energie tvoří řádově $10^{-9}$ hmotnosti molekuly. Protože byla chyba $B$ řádově $10^{-4}$, šlo o oprávněnou aproximaci.


\section{Závěr}
Bylo to hezké. assadfasd

\section{Literatura}
\bibliography{literatura} 
\bibliographystyle{ieeetr}
 
\end{document}