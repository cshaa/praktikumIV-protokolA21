% !TEX program = xelatex

% Nejlepší zážitek zaručí:
%
% TeX distribuce: texlive-full
%
% Editor:
%   VS Code s doplňky
%       * LaTeX Workshop
%       * LaTeX Utilities
%       * Gnuplot
%
% Další závislosti:
%   latexmk
%   bibtex
%   gnuplot


% Jak používat:
% Zkompilovat: make
% Gnuplot: make gnuplot
% Vyčistit: make clean


% Základní balíčky
\documentclass[10pt,a4paper]{article}
\usepackage[utf8]{inputenc}
\usepackage[czech]{babel}
\usepackage{graphicx}
\usepackage{lmodern}
\usepackage{amsmath}
\usepackage{hyperref}
\usepackage{gensymb}
\usepackage[top = 1.5cm, bottom = 1.5cm, left = 1.5cm, right = 1.5cm]{geometry}

% Bibtex
\usepackage{etoolbox}
\patchcmd{\thebibliography}{\section*{\refname}}{}{}{}

% Pro titulní stránku
\usepackage{titlesec}
\usepackage{setspace}
\usepackage{framed}
\usepackage{array}

% Vlastní balíčky 
\usepackage{gnuplottex}
\usepackage{epstopdf}
\usepackage{csvsimple}
\usepackage{units}


\renewcommand{\U}[1]{\ensuremath{\,\mathrm{#1}}}
\newcommand{\°}{\degree}

\newcommand{\titjmeno}{Michal Grňo}
\newcommand{\titobor}{FOF}


\newcommand{\titcislo}{A21}
\newcommand{\titnazev}{Studium rentgenových spekter}
\newcommand{\titmereni}{11. 11. 2019}
\newcommand{\titodevzdani}{17. 11. 2019}


\begin{document}


\thispagestyle{empty}
\newgeometry{top = 2.5cm, bottom = 0cm, left = 2.5cm, right = 3cm}

{%T tomto je uzavřena celá titulka
%Tloušťka rámečku
\setlength{\fboxrule}{1.5pt}

\noindent
\framebox{
\begin{minipage}{\textwidth}
\setlength{\parindent}{17.62482 pt}
\phantom{d}

\begin{minipage}{0.6\textwidth}
{
\Large Kabinet výuky obecné fyziky, UK MFF\\
}
\vspace*{0.2cm}

{
\bfseries
\huge Fyzikální praktikum %ČÍSLO
}
\end{minipage}
\begin{minipage}{0.4\textwidth}
\begin{center}
\includegraphics[width=4.5cm]{ZFP.jpg}
\end{center}
\end{minipage}\\\\

%\vspace*{0.5cm}

{
\setstretch{1.5}
\Large
\noindent
Úloha č. \titcislo

\noindent
Název úlohy: \titnazev

\noindent
Jméno: \titjmeno
\hspace*{\fill}
Obor: \titobor

\noindent
Datum měření: \titmereni
\hspace*{\fill}
Datum odevzdání: \titodevzdani

\phantom{d}
}
\end{minipage}
}
%Konec horního rámečku

{
\phantom{d}

\Large
Připomínky opravujícího:\\
\vspace*{6.75cm}
}

\newcommand{\linka}{\noalign{\hrule height 1pt}}
\newcommand{\linkadva}{\noalign{\hrule height 1.5pt}}
\setlength\extrarowheight{9.5pt}
\Large
\noindent
\begin{tabular}{!{\vrule width 1.5pt} l !{\vrule width 1pt} c !{\vrule width 1pt} c !{\vrule width 1.5pt}}
\linkadva
   & Možný počet bodů & Udělený počet bodů \\\linkadva
  Práce při měření & 0-3 &  \\\linka
  Teoretická část & 0-2 &  \\\linka
  Výsledky a zpracování měření & 0-9 &  \\\linka
  Diskuse výsledků & 0-4 &  \\\linka
  Závěr & 0-1 &  \\\linka
  Použitá literatura & 0-1 &  \\\linkadva
  \hspace*{\fill} \textbf{Celkem} \hspace*{\fill}& max. 20 &  \\
\linkadva
\end{tabular}
\phantom{d}

Posuzoval: \hspace*{\fill}dne:~~~~~~~~~~~~~~~~~

}%Konec uzavření titulky
\newpage
\newgeometry{top = 2cm, bottom = 2cm, left = 2cm, right = 2cm}
\setcounter{page}{1}

\section{Pracovní úkoly}
\begin{enumerate}
    \item S využitím krystalu LiF jako analyzátoru proveďte měření následujících rentgenových spekter:
    \begin{enumerate}
        \item Rentgenka s Cu anodou.
        \begin{enumerate}
            \item proměřte krátkovlnné oblasti spekter brzdného záření při napětích 15 kV/1 mA, 25 kV/0,8 mA, 30 kV/0,8 mA, 33 kV/0,8 mA. K měření používejte tyto parametry: clonu o průměru 2 mm, interval Braggova úhlu pro 15 kV v rozmezí (10° – 15°) s krokem 0.2° a dobou expozice 8 s a pro ostatní napětí interval Braggova úhlu (3° – 10°) s krokem 0.2° a dobou expozice 5 s;
            \item proměřte charakteristická spektra rentgenky při napětích 15 kV/1 mA a 33 kV/0,8 mA. K měření používejte tyto parametry: clonu o průměru 2 mm, interval Braggova úhlu (15° – 30°), krok 0.1° a dobu expozice 2 s;
            \item proměřte tvar spektra s Zr absorbérem. K měření používejte tyto parametry: clonu s Zr absorbérem tloušťky 0.05 mm, interval Braggova úhlu (3° – 30°), krok 0.1° a dobu expozice 2 s;
            \item proměřte tvar spektra s Ni absorbérem. K měření používejte tyto parametry: clonu s Ni absorbérem tloušťky 0.01 mm, interval Braggova úhlu (3° – 30°), krok 0.1° a dobu expozice 2 s.
        \end{enumerate}
        \item Rentgenka s Fe anodou
        \begin{enumerate}
            \item proměřte charakteristické spektrum rentgenky při napětí 33 kV/0.8 mA. K měření používejte tyto parametry: clonu o průměru 2 mm, interval Braggova úhlu (3° – 30°), krok 0.1° a dobu expozice 2 s;
            \item proměřte tvar spektra s Zr absorbérem. K měření používejte tyto parametry: clonu s Zr absorbérem tloušťky 0.05 mm, interval Braggova úhlu (3° – 30°), krok 0.1° a dobu expozice 3 s.
        \end{enumerate}
        \item Rentgenka s Mo anodou.
        \begin{enumerate}
            \item proměřte charakteristické spektrum rentgenky při napětí 33 kV/0.8 mA. K měření používejte tyto parametry: clonu o průměru 2  mm, interval Braggova úhlu (3° – 35°), krok 0.1° a dobu expozice 3 s.
        \end{enumerate}
        \item Rentgenka s Cu anodou:
        \begin{enumerate}
            \item proměřte charakteristické spektrum rentgenky při napětí 33 kV/0.8 mA v intervalu Braggova úhlu (42° – 51°). K měření používejte tyto parametry: clonu o průměru 2 mm, krok 0.1° a dobou expozice 2 s.
        \end{enumerate}
    \end{enumerate}
    \item Interpretujte naměřené výsledky (pro mezirovinnou vzdálenost krystalu LiF používejte hodnotu d = 201,4 pm):
    \begin{enumerate}
        \item Krátkovlnná mez brzdného záření
        \begin{enumerate}
            \item Ze změřených mezních vlnových délek (respektive frekvencí) určete hodnotu Planckovy konstanty a oceňte přesnost měření
        \end{enumerate}
        \item Moseleyův zákon
        \begin{enumerate}
            \item Přesvědčte se, že naměřené úhlové frekvence spektrálních čar Kα a Kβ pro různé prvky splňují Moseleyův zákon. Ze směrnice příslušné závislosti určete hodnotu Rydbergovy úhlové frekvence a využitím této hodnoty určete též průměrnou hodnotu stínící konstanty.
            \item Přesvědčte se, že i naměřené polohy absorpčních hran Zr a Ni splňují Moseleyův zákon.
            \item Všimněte si, že absorpční hrana Ni koinciduje se spektrální čarou Kβ mědi; této skutečnosti se využívá v rentgenové difraktografii pro monochromatizaci charakteristického spektra mědi. Z provedeného měření určete filtrační efekt niklu pro čáru Kβ.
        \end{enumerate}
        \item Úhlová disperze
        \begin{enumerate}
            \item Ze změřených spekter molybdenu určete velikost úhlové disperze pro různé řády difrakce.
        \end{enumerate}
    \end{enumerate}
\end{enumerate}


\section{Teoretická část}


\section{Výsledky měření}


\section{Diskuse}


\section{Závěr}



\section{Literatura}
\bibliography{literatura} 
\bibliographystyle{ieeetr}
 
\end{document}