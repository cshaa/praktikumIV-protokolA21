% !TEX program = xelatex

% Nejlepší zážitek zaručí:
%
% TeX distribuce: texlive-full
%
% Editor:
%   VS Code s doplňky
%       * LaTeX Workshop
%       * LaTeX Utilities
%       * Gnuplot
%
% Další závislosti:
%   latexmk
%   bibtex
%   gnuplot


% Jak používat:
% Zkompilovat: make
% Gnuplot: make gnuplot
% Vyčistit: make clean


% Základní balíčky
\documentclass[10pt,a4paper]{article}
\usepackage[utf8]{inputenc}
\usepackage[czech]{babel}
\usepackage{graphicx}
\usepackage{lmodern}
\usepackage{amsmath}
\usepackage{hyperref}
\usepackage{gensymb}
\usepackage[top = 1.5cm, bottom = 1.5cm, left = 1.5cm, right = 1.5cm]{geometry}

% Bibtex
\usepackage{etoolbox}
\patchcmd{\thebibliography}{\section*{\refname}}{}{}{}

% Pro titulní stránku
\usepackage{titlesec}
\usepackage{setspace}
\usepackage{framed}
\usepackage{array}

% Vlastní balíčky 
\usepackage{gnuplottex}
\usepackage{epstopdf}
\usepackage{csvsimple}
\usepackage{units}


\renewcommand{\U}[1]{\ensuremath{\,\mathrm{#1}}}
\newcommand{\°}{\degree}

\newcommand{\titjmeno}{Michal Grňo}
\newcommand{\titobor}{FOF}


\newcommand{\titcislo}{A21}
\newcommand{\titnazev}{Studium rentgenových spekter}
\newcommand{\titmereni}{11. 11. 2019}
\newcommand{\titodevzdani}{17. 11. 2019}


\begin{document}


\thispagestyle{empty}
\newgeometry{top = 2.5cm, bottom = 0cm, left = 2.5cm, right = 3cm}

{%T tomto je uzavřena celá titulka
%Tloušťka rámečku
\setlength{\fboxrule}{1.5pt}

\noindent
\framebox{
\begin{minipage}{\textwidth}
\setlength{\parindent}{17.62482 pt}
\phantom{d}

\begin{minipage}{0.6\textwidth}
{
\Large Kabinet výuky obecné fyziky, UK MFF\\
}
\vspace*{0.2cm}

{
\bfseries
\huge Fyzikální praktikum %ČÍSLO
}
\end{minipage}
\begin{minipage}{0.4\textwidth}
\begin{center}
\includegraphics[width=4.5cm]{ZFP.jpg}
\end{center}
\end{minipage}\\\\

%\vspace*{0.5cm}

{
\setstretch{1.5}
\Large
\noindent
Úloha č. \titcislo

\noindent
Název úlohy: \titnazev

\noindent
Jméno: \titjmeno
\hspace*{\fill}
Obor: \titobor

\noindent
Datum měření: \titmereni
\hspace*{\fill}
Datum odevzdání: \titodevzdani

\phantom{d}
}
\end{minipage}
}
%Konec horního rámečku

{
\phantom{d}

\Large
Připomínky opravujícího:\\
\vspace*{6.75cm}
}

\newcommand{\linka}{\noalign{\hrule height 1pt}}
\newcommand{\linkadva}{\noalign{\hrule height 1.5pt}}
\setlength\extrarowheight{9.5pt}
\Large
\noindent
\begin{tabular}{!{\vrule width 1.5pt} l !{\vrule width 1pt} c !{\vrule width 1pt} c !{\vrule width 1.5pt}}
\linkadva
   & Možný počet bodů & Udělený počet bodů \\\linkadva
  Práce při měření & 0-3 &  \\\linka
  Teoretická část & 0-2 &  \\\linka
  Výsledky a zpracování měření & 0-9 &  \\\linka
  Diskuse výsledků & 0-4 &  \\\linka
  Závěr & 0-1 &  \\\linka
  Použitá literatura & 0-1 &  \\\linkadva
  \hspace*{\fill} \textbf{Celkem} \hspace*{\fill}& max. 20 &  \\
\linkadva
\end{tabular}
\phantom{d}

Posuzoval: \hspace*{\fill}dne:~~~~~~~~~~~~~~~~~

}%Konec uzavření titulky
\newpage
\newgeometry{top = 2cm, bottom = 2cm, left = 2cm, right = 2cm}
\setcounter{page}{1}

\section{Pracovní úkoly}
\begin{enumerate}
    \item S využitím krystalu LiF jako analyzátoru proveďte měření následujících rentgenových spekter:
    \begin{enumerate}
        \item Rentgenka s Cu anodou.
        \begin{enumerate}
            \item proměřte krátkovlnné oblasti spekter brzdného záření při napětích 15 kV/1 mA, 25 kV/0,8 mA, 30 kV/0,8 mA, 33 kV/0,8 mA. K měření používejte tyto parametry: clonu o průměru 2 mm, interval Braggova úhlu pro 15 kV v rozmezí (10° – 15°) s krokem 0.2° a dobou expozice 8 s a pro ostatní napětí interval Braggova úhlu (3° – 10°) s krokem 0.2° a dobou expozice 5 s;
            \item proměřte charakteristická spektra rentgenky při napětích 15 kV/1 mA a 33 kV/0,8 mA. K měření používejte tyto parametry: clonu o průměru 2 mm, interval Braggova úhlu (15° – 30°), krok 0.1° a dobu expozice 2 s;
            \item proměřte tvar spektra s Zr absorbérem. K měření používejte tyto parametry: clonu s Zr absorbérem tloušťky 0.05 mm, interval Braggova úhlu (3° – 30°), krok 0.1° a dobu expozice 2 s;
            \item proměřte tvar spektra s Ni absorbérem. K měření používejte tyto parametry: clonu s Ni absorbérem tloušťky 0.01 mm, interval Braggova úhlu (3° – 30°), krok 0.1° a dobu expozice 2 s.
        \end{enumerate}
        \item Rentgenka s Fe anodou
        \begin{enumerate}
            \item proměřte charakteristické spektrum rentgenky při napětí 33 kV/0.8 mA. K měření používejte tyto parametry: clonu o průměru 2 mm, interval Braggova úhlu (3° – 30°), krok 0.1° a dobu expozice 2 s;
            \item proměřte tvar spektra s Zr absorbérem. K měření používejte tyto parametry: clonu s Zr absorbérem tloušťky 0.05 mm, interval Braggova úhlu (3° – 30°), krok 0.1° a dobu expozice 3 s.
        \end{enumerate}
        \item Rentgenka s Mo anodou.
        \begin{enumerate}
            \item proměřte charakteristické spektrum rentgenky při napětí 33 kV/0.8 mA. K měření používejte tyto parametry: clonu o průměru 2  mm, interval Braggova úhlu (3° – 35°), krok 0.1° a dobu expozice 3 s.
        \end{enumerate}
        \item Rentgenka s Cu anodou:
        \begin{enumerate}
            \item proměřte charakteristické spektrum rentgenky při napětí 33 kV/0.8 mA v intervalu Braggova úhlu (42° – 51°). K měření používejte tyto parametry: clonu o průměru 2 mm, krok 0.1° a dobou expozice 2 s.
        \end{enumerate}
    \end{enumerate}
    \item Interpretujte naměřené výsledky (pro mezirovinnou vzdálenost krystalu LiF používejte hodnotu d = 201,4 pm):
    \begin{enumerate}
        \item Krátkovlnná mez brzdného záření
        \begin{enumerate}
            \item Ze změřených mezních vlnových délek (respektive frekvencí) určete hodnotu Planckovy konstanty a oceňte přesnost měření
        \end{enumerate}
        \item Moseleyův zákon
        \begin{enumerate}
            \item Přesvědčte se, že naměřené úhlové frekvence spektrálních čar $K_\alpha$ a $K_\beta$ pro různé prvky splňují Moseleyův zákon. Ze směrnice příslušné závislosti určete hodnotu Rydbergovy úhlové frekvence a využitím této hodnoty určete též průměrnou hodnotu stínící konstanty.
            \item Přesvědčte se, že i naměřené polohy absorpčních hran Zr a Ni splňují Moseleyův zákon.
            \item Všimněte si, že absorpční hrana Ni koinciduje se spektrální čarou $K_\beta$ mědi; této skutečnosti se využívá v rentgenové difraktografii pro monochromatizaci charakteristického spektra mědi. Z provedeného měření určete filtrační efekt niklu pro čáru $K_\beta$.
        \end{enumerate}
        \item Úhlová disperze
        \begin{enumerate}
            \item Ze změřených spekter molybdenu určete velikost úhlové disperze pro různé řády difrakce.
        \end{enumerate}
    \end{enumerate}
\end{enumerate}


\section{Teoretická část}
Rentgenka je zařízení, které vyzařuje rentgenové záření, pokud mu dodáváme dostatečné napětí. Je tvořena vakuovou baňkou, uvnitř které se nachází žhavená katoda, z níž vylétávají elektrony, a anoda, na kterou dopadají a při dopadu vyzařují elektromagnetické záření v rentgenové oblasti. Vznikající záření má dvě složky odpovídající dvěma různým způsobům, kterými vzniká. Jednak spojité brzdné záření, které vzniká když elektron prudce brzdí v elektromagnetickém poli anody. Toto záření není závislé na materiálu anody a nejvíce energetický foton, který dokáže při daném napětí vyprodukovat, má vlnovou délku $\lambda_m$, pro kterou platí
\begin{equation}
    eU = \frac{hc}{\lambda_m}.
    \label{mezni_lambda}
\end{equation}

My budeme provádět spektroskopii pomocí difrakce na mřížce LiF, vztah mezi měřeným úhlem a odpovídající vlnovou délkou udává tzv. Braggova rovnice:
\begin{equation}
    2d \sin \varphi = n\lambda,
    \label{bragg}
\end{equation}
kde $d=201.4 \U{pm}$ je mřížková konstanta LiF a $n$ je celé číslo udávající řád difrakčního maxima. Víme, že naměřený úhel v našem zařízení je zatížený aditivní chybou, z měření tedy budeme mít úhly $\vartheta$, pro které platí $\varphi = \vartheta + \vartheta_0$.

Budeme chtít pro ověření z naměřených dat vypočítat Planckovu konstatnu, vyjádříme si ji z rovnic~\eqref{mezni_lambda}~a~\eqref{bragg}, pro mezní úhel $\varphi_m$ potom bude platit:
\begin{align}
    h &= \frac{2eUd}{c} \frac{\sin \varphi_m}{n}, &
    \Delta h &= \frac{2eUd}{c} \frac{\cos \varphi_m}{n} \; \Delta\varphi
    \label{planck}
\end{align}


Druhý typ záření, který rentgenka produkuje, je tzv. charakteristické záření, které vzniká při excitaci a následné deexcitaci atomu anody. Toto záření je závislé na materiálu anody, úhlová frekvence fotonu odpovídající přechodu z $m$-tého excitovaného stavu do $n$-tého stavu je podle Rydbergova vztahu
\begin{equation}
    \omega = R_\omega (Z - s)^2 \left( \frac{1}{n^2} - \frac{1}{m^2} \right),
\end{equation}
kde $Z$ je atomové číslo prvku anody, $s$ je stínící konstanta a pro $R_\omega$ platí
\begin{equation}
    R_\omega = \frac{m_e e^4}{32 \pi^2 {\varepsilon_0}^2 \hbar^3}
    \approx  2.0606 \cdot 10^{16} \U{s^{-1}}
\end{equation}
Nás budou zajímat především spektrální čáry $K_\alpha$ a $K_\beta$, které odpovídají pádu z $m=2$, resp. $m=3$ do základního stavu $n=1$. Dosazením $m,n$ dostaneme tzv. Moseleyův zákon, který určuje vztah $\omega$ a $Z$:
\begin{align}
    \sqrt{\omega(K_\alpha)} &= \frac{\sqrt{3R_\omega}}{2}(Z-s),
    \label{omega-k-alpha} \\
    \sqrt{\omega(K_\beta)}  &= \frac{\sqrt{8R_\omega}}{3}(Z-s).
    \label{omega-k-beta}
\end{align}
Převedeme-li vlnovou délku v rovnici \eqref{bragg} na úhlovou frekvenci a dosadíme-li do předchozích rovnic, dostaneme lineární vztah:
\begin{align}
    \sqrt{\frac{n}{\sin\varphi(K_\alpha)}}
    = \frac{1}{2}\sqrt{ \frac{3R_\omega d}{\pi c} }(Z - s),
    \label{linearni-k-alpha} \\
    \sqrt{\frac{n}{\sin\varphi(K_\beta)}}
    = \frac{1}{3}\sqrt{ \frac{8R_\omega d}{\pi c} }(Z - s).
    \label{linearni-k-beta}
\end{align}
To jsou lineární vztahy tvaru $y = A(x - B)$, ze kterých můžeme lineární regresí zjistit stínící konstantu $s=B$ a Rydbergovu konstantu $R_\omega = \tfrac{4 \pi c}{3 d} A^2$, resp. $\tfrac{9 \pi c}{8 d} A^2$.


\section{Výsledky měření}
Nejprve jsme měřili brzdné záření na rentgence s měděnou anodou. Extrapolací z grafů jsme určili mezní úhly:
\begin{table}[h!]
    \centering
    \begin{tabular}{ r|rl }
        \bfseries $U [\U{kV}]$ &
        \multicolumn{2}{c}{$\vartheta [\°]$}
        \csvreader[ head to column names ]{data_brzdne.csv}{}
        {
            \csviffirstrow{\\\hline}{\\}
            \valU & \valtheta & $\pm$ \valthetaerr
        }
    \end{tabular}
    \caption{Mezní úhly $\vartheta$}
    \label{mezni-uhly}
\end{table}

\begin{figure}[p]
    \centering
    \begin{gnuplot}[terminal=epslatex,terminaloptions={color size 18cm, 8cm}]

        set datafile separator ','
        load 'constants.cfg'

        LF_File = "data_brzdne.csv"
        LF_Columns = 3
        load 'loadfile.cfg'

        N = LF_Rows - 1
        array U[N]
        array theta[N]
        array thetaerr[N]
        do for [i = 1:N] {
            U[i]        = LF_Col1[i+1]*1000
            theta[i]    = LF_Col2[i+1]
            thetaerr[i] = LF_Col3[i+1]
        }

        array h[N]
        array herr[N]
        do for [i = 1:N] {
            const = 2*e*U[i]*(d/c)
            h[i] = const * sin(theta[i]*deg)
            herr[i] = const * cos(theta[i]*deg) * thetaerr[i]*deg
        }

        set print "data_planck.csv.tmp"
        print "valU,valh,valherr"
        do for [i = 1:N] {
            print sprintf("%d,%f,%f", U[i]/1000, h[i]*1e34, herr[i]*1e34)
        }

        A = 1
        f(x) = A
        hh(x) = hh*1e34
        fit f(x) 'data_planck.csv.tmp' skip 1 using 1:2:3 yerrors via A

        set xrange [12:36]
        set yrange [5.6:6.8]
        set xlabel "$\theta [\°]$"
        set ylabel "$h [\U{10^{-34} J}]$"
        plot sample [i=1:N] '+' using (U[i]/1000):(h[i]*1e34):(herr[i]*1e34) w yerrorbars t "data", f(x) t "vážený průměr", hh(x) t "skutečná hodnota"

    \end{gnuplot}
    \caption{Naměřené hodnoty Planckovy konstanty}
    \label{graf-planck}
\end{figure}

Připomínáme, že $\vartheta$ značíme úhel ještě před korekcí na systematickou chybu. Úhel po korekci značíme $\varphi = \vartheta + \vartheta_0$.

Hodnoty Planckovy konstanty vypočtené podle \eqref{planck}, jejich průměř\footnote{Průměr je vážený převráceným čtvercem chyby.} a porovnání se skutečnou hodnotou je v grafu č. \ref{graf-planck}. Vidíme, že se skutečná hodnota signifikantně liší od té naměřené – to protože jsme zatím předpokládali, že systematická chyba $\vartheta_0 = 0$. Numericky nyní vyřešíme, pro jakou hodnotu $\vartheta_0$ se budou skutečná hodnota $h$ a vážený průměr rovnat. Získáme tím
\begin{equation}
    \vartheta_0 = 0.574 \°.
    \label{systematicka_chyba}
\end{equation}

Následně jsme měřili charakteristická spektra pro různé materiály anod. Pozorovali jsme maxima $n$-tého řádu na těchto úhlech:

\begin{table}[h!]
    \centering
    \begin{tabular}{ c|c|c|r|r }
        \multicolumn{1}{c|}{prvek} &
        \multicolumn{1}{c|}{$n$} & 
        \multicolumn{1}{c|}{$U [\U{kV}]$} &
        \multicolumn{1}{c|}{$\theta(K_\alpha) [\°]$} &
        \multicolumn{1}{c}{$\theta(K_\beta) [\°]$}
        \csvreader[ head to column names ]{data_charakt.csv}{}
        {
            \csviffirstrow{\\\hline}{\\}
            $^{\valZ}$\prvek &
            \valn & \valU &
            \thetaalpha &
            \thetabeta
        }
    \end{tabular}
    \caption{Úhly $\vartheta$ maxim charakteristického záření}
    \label{charakt-uhly}
\end{table}

\begin{figure}[p]
    \centering
    \begin{gnuplot}[terminal=epslatex,terminaloptions={color size 18cm, 8cm}]
        
        set datafile separator ','
        load 'constants.cfg'

        LF_File = "data_charakt.csv"
        LF_Columns = 6
        load 'loadfile.cfg'

        y(n, theta) = sqrt(n/sin(theta*deg))

        set xrange [24:44]
        set yrange [1.2:2.8]
        plot 'data_charakt.csv' skip 1 using 5:(y($1,$3)) notitle

    \end{gnuplot}
    \caption{Lineární závislost z \eqref{linearni-k-alpha} a \eqref{linearni-k-beta}}
    \label{graf-linearni}
\end{figure}



\section{Diskuse}


\section{Závěr}



\section{Literatura}
\bibliography{literatura} 
\bibliographystyle{ieeetr}
 
\end{document}